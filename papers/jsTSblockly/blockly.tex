\section{Blockly Overview}
\label{sec:blockly}

Not suprisingly, the core abstraction of Blockly is the \emph{Block},
which is used to represent statements, expressions, values and variables.  
Blocks have \emph{connectors} that allow them to be sequenced
horizontally or vertically in space.  In the default Blockly
layout, vertically sequenced blocks
represent program statements, while horizontally
sequenced blocks represent program expressions/values.
Blockly also supports the nesting of blocks, and so a collection of
blocks naturally represents an abstract syntax tree. 

While Blockly does attach a loose meaning to blocks,
their actual semantics given to blocks is entirely up to the Blockly developer.
We use the TypeScript language (and its type system) to 
give a more precise meaning to blocks. 

Block connectors are categorized and constrained as follows:
\begin{itemize}

\item \emph{previous and next connectors}, both optional, allow a block to be vertically sequenced - Blockly
  enforces that a block with a previous connector cannnot have an output connector; according
  to Blockly documentation, ``a statement block will usually have both a previous connection and 
  a next connection'';

\item a block may have a single \emph{output connector}, 
      which appears as a male jigsaw connector on a block's left-side; according to 
      Blockly documentation, ``Blocks with an output are usually called value blocks'';

\item a block may have multiple \emph{inputs}, which can appears as holes in a block
      or female jigsaw connectors on a block's right-side, about which we'll say more below. 
\end{itemize}

Block inputs come in three basic forms:
\begin{enumerate}
  \item \emph{fields}, which represent terminals (constants, literals, variables);
  \item \emph{value inputs} (receives value from output block) - value inputs
      and value blocks allow one to create expression trees;
  \item \emph{statement inputs}, allow one to create statement trees;
\end{enumerate}

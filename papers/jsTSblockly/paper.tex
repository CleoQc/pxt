\documentclass[sigplan,10pt]{acmart}
%\settopmatter{printfolios=false,printccs=false,printacmref=false}
\usepackage{graphicx}
\usepackage{listings}
\usepackage{enumitem}
\usepackage{hyperref}
\newcommand{\MC}{MakeCode\ }
\newcommand{\MCN}{MakeCode}

\setlist[itemize]{leftmargin=*}
\lstset{ %
language=C++,                % choose the language of the code
basicstyle=\footnotesize,       % the size of the fonts that are used for the code
numbers=left,                   % where to put the line-numbers
numberstyle=\footnotesize,      % the size of the fonts that are used for the line-numbers
stepnumber=1,                   % the step between two line-numbers. If it is 1 each line will be numbered
numbersep=5pt,                  % how far the line-numbers are from the code
backgroundcolor=\color{white},  % choose the background color. You must add \usepackage{color}
showspaces=false,               % show spaces adding particular underscores
showstringspaces=false,         % underline spaces within strings
showtabs=false,                 % show tabs within strings adding particular underscores
frame=single,           % adds a frame around the code
tabsize=2,          % sets default tabsize to 2 spaces
captionpos=b,           % sets the caption-position to bottom
breaklines=true,        % sets automatic line breaking
breakatwhitespace=false,    % sets if automatic breaks should only happen at whitespace
escapeinside={\%*}{*)}          % if you want to add a comment within your code
}

%% For double-blind review submission, w/ CCS and ACM Reference
%\documentclass[sigplan,10pt,review,anonymous]{acmart}\settopmatter{printfolios=true}
%% For single-blind review submission, w/o CCS and ACM Reference (max submission space)
%\documentclass[sigplan,10pt,review]{acmart}\settopmatter{printfolios=true,printccs=false,printacmref=false}
%% For single-blind review submission, w/ CCS and ACM Reference
%\documentclass[sigplan,10pt,review]{acmart}\settopmatter{printfolios=true}
%% For final camera-ready submission, w/ required CCS and ACM Reference
%\documentclass[sigplan,10pt]{acmart}\settopmatter{}


%% Conference information
%% Supplied to authors by publisher for camera-ready submission;
%% use defaults for review submission.
% \acmConference[PL'17]{ACM SIGPLAN Conference on Programming Languages}{January 01--03, 2017}{New York, NY, USA}
% \acmYear{2017}
% \acmISBN{} % \acmISBN{978-x-xxxx-xxxx-x/YY/MM}
% \acmDOI{} % \acmDOI{10.1145/nnnnnnn.nnnnnnn}
\startPage{1}

%% Copyright information
%% Supplied to authors (based on authors' rights management selection;
%% see authors.acm.org) by publisher for camera-ready submission;
%% use 'none' for review submission.
\setcopyright{none}
%\setcopyright{acmcopyright}
%\setcopyright{acmlicensed}
%\setcopyright{rightsretained}
%\copyrightyear{2017}           %% If different from \acmYear

%% Bibliography style
\bibliographystyle{ACM-Reference-Format}
%% Citation style
%\citestyle{acmauthoryear}  %% For author/year citations
%\citestyle{acmnumeric}     %% For numeric citations
%\setcitestyle{nosort}      %% With 'acmnumeric', to disable automatic
                            %% sorting of references within a single citation;
                            %% e.g., \cite{Smith99,Carpenter05,Baker12}
                            %% rendered as [14,5,2] rather than [2,5,14].
%\setcitesyle{nocompress}   %% With 'acmnumeric', to disable automatic
                            %% compression of sequential references within a
                            %% single citation;
                            %% e.g., \cite{Baker12,Baker14,Baker16}
                            %% rendered as [2,3,4] rather than [2-4].


%%%%%%%%%%%%%%%%%%%%%%%%%%%%%%%%%%%%%%%%%%%%%%%%%%%%%%%%%%%%%%%%%%%%%%
%% Note: Authors migrating a paper from traditional SIGPLAN
%% proceedings format to PACMPL format must update the
%% '\documentclass' and topmatter commands above; see
%% 'acmart-pacmpl-template.tex'.
%%%%%%%%%%%%%%%%%%%%%%%%%%%%%%%%%%%%%%%%%%%%%%%%%%%%%%%%%%%%%%%%%%%%%%


%% Some recommended packages.
\usepackage{booktabs}   %% For formal tables:
                        %% http://ctan.org/pkg/booktabs
\usepackage{subcaption} %% For complex figures with subfigures/subcaptions
                        %% http://ctan.org/pkg/subcaption

\usepackage{courier}

\begin{document}

%% Title information
\title{TypeScript: From JavaScript \\ to Blockly and Back}         %% [Short Title] is optional;
                                        %% when present, will be used in
                                        %% header instead of Full Title.
\subtitle{Microsoft MakeCode Team}                     %% \subtitle is optional


%% Author information
%% Contents and number of authors suppressed with 'anonymous'.
%% Each author should be introduced by \author, followed by
%% \authornote (optional), \orcid (optional), \affiliation, and
%% \email.
%% An author may have multiple affiliations and/or emails; repeat the
%% appropriate command.
%% Many elements are not rendered, but should be provided for metadata
%% extraction tools.

% %% Author with single affiliation.
% \author{James Devine}
% \affiliation{
%   \institution{Lancaster University, UK}            %% \institution is required
% }
% \email{james@devine.eu}  
% \author{Joe Finney}
% \affiliation{
%   \institution{Lancaster University, UK}            %% \institution is required
% }
% \email{j.finney@lancaster.ac.uk} 
% \author{Micha\l Moskal}
% \affiliation{
%   \institution{Microsoft, USA}            %% \institution is required
% }
% \email{mmoskal@microsoft.com} 
% \author{Peli de Halleux}
% \affiliation{
%   \institution{Microsoft, USA}            %% \institution is required
% }
% \email{jhalleux@microsoft.com} 
% \author{Thomas Ball}
% \affiliation{
%   \institution{Microsoft, USA}            %% \institution is required
% }
% \email{tball@microsoft.com} 
% \author{Steve Hodges}
% \affiliation{
%   \institution{Microsoft, UK}            %% \institution is required
% }
% \email{shodges@microsoft.com} 

%% Author with two affiliations and emails.
% \author{First2 Last2}
% \authornote{with author2 note}          %% \authornote is optional;
%                                         %% can be repeated if necessary
% \orcid{nnnn-nnnn-nnnn-nnnn}             %% \orcid is optional
% \affiliation{
%   \position{Position2a}
%   \department{Department2a}             %% \department is recommended
%   \institution{Institution2a}           %% \institution is required
%   \streetaddress{Street2a Address2a}
%   \city{City2a}
%   \state{State2a}
%   \postcode{Post-Code2a}
%   \country{Country2a}                   %% \country is recommended
% }
% \email{first2.last2@inst2a.com}         %% \email is recommended


%% Abstract
%% Note: \begin{abstract}...\end{abstract} environment must come
%% before \maketitle command
% \begin{abstract}
    While there are many JavaScript libraries for building solutions for a 
    wide range of problems, it's not easy for novices to harness their power.  
    We show how TypeScript, a gradually typed superset of JavaScript, can be used
    to bridge the jap between JavaScript and Blockly, a framework for creating
    block-based programming environments that
    greatly reduces the potential for syntax and semantic errors. 
    In particular, we define a mapping from TypeScript to 
    Blockly that makes it simple
    to create a domain-specific Blockly editor for a JavaScript library via a 
    TypeScript declaration file.  This mapping is support by the \MC
    framework (\url{www.makecode.com}). We present a set of \MC environments
    developed by graduate students at XYZ. 
\end{abstract}

\begin{abstract}
    While there are many JavaScript libraries for building solutions for a 
    wide range of problems, it's not easy for novices to harness their power.  
    We show how TypeScript, a gradually typed superset of JavaScript, can be used
    to bridge the jap between JavaScript and Blockly, a framework for creating
    block-based programming environments that
    greatly reduces the potential for syntax and semantic errors. 
    In particular, we define a mapping from TypeScript to 
    Blockly that makes it simple
    to create a domain-specific Blockly editor for a JavaScript library via a 
    TypeScript declaration file.  This mapping is support by the \MC
    framework (\url{www.makecode.com}). We present a set of \MC environments
    developed by graduate students at XYZ. 
\end{abstract}


%% Keywords
%% comma separated list
%\keywords{keyword1, keyword2, keyword3}  %% \keywords are mandatory in final camera-ready submission


%% \maketitle
%% Note: \maketitle command must come after title commands, author
%% commands, abstract environment, Computing Classification System
%% environment and commands, and keywords command.
\maketitle

\section{Introduction}

% set up the context and constraints
% we are looking at an easy on-ramp to coding on the web for complete beginners
% leverage JavaScript and TypeScript

Introducing programming to beginners used to be a process fraught with
accidental complexity, due to the need to install tool chains and IDEs, 
typically created with the experienced developer in mind. Today, 
the web has made available a plethora of programming environments;
most any programming language can be experienced via the web. 
Some popular examples include:
\begin{itemize}
\item {\bf JavaScript}: \url{https://jsfiddle.net/};
\item {\bf Python}: \url{https://www.learnpython.org};
\item {\bf Ruby}: \url{http://tryruby.org}.
\end{itemize}
Not surprisingly, all the above approaches still use text editors for coding;
this means that the possibility for introducing errors is still great, especially 
for beginners who don't understand the language. Features such as step-by-step 
tutorials, auto-fixing, and intellisense can
improve the experience, but the starting point still is not the most welcoming. 

% going larger

The ``Hour of Code'' experience,\footnote{\url{https://hourofcode.com/learn}} 
has introduced over three hundred million students to coding,\footnote{https://code.org/about/2016} 
with the express 
purpose of demystifying and breaking stereotypes about coding.
In order to reach this many people, \url{code.org} used
Google's Blockly~\cite{Blocky2015}~\footnote{\url{www.github.com/google/blockly}},
a JavaScript framework inspired by MIT's Scratch programming 
environment~\cite{ScratchCACM2009},
to provide a very simple beginning programming experience, 
free of the possibility of syntax errors. 

\begin{figure}[t]
    \includegraphics[width=\columnwidth]{pics/hourofcode}
\caption{\label{fig:hoc}Hour of Code example.}
\end{figure}

Figure~\ref{fig:hoc} shows a screen snapshot from one of the many 
``Hour of Code'' tutorials, in the family of ``maze solving'' problems.
In this example, the goal is to program the Minecraft avatar ``Steve''
on the left to move to the sheep on the right. To create a program,
the user selects from a simple palette of three blocks
(shown under the ``Blocks'' heading) and drags these blocks into
the workspace. In this very simple example, the
program is simply a sequence of commands. In later steps, 
concepts such as loops and XYZ are introduced.

In Blockly, visual blocks represent structured control-flow statements such as loops 
and if-then-else conditionals, as well as program expressions, values and variables. 
Blocks also represent function calls to domain-specific APIS, via Blockly's support for \emph{custom 
blocks}.  In the case of Figure~\ref{fig:hoc}, the domain-specific APIs are the commands
for moving ``Steve''.

% Blockly can be compiled to a variety of languages, the main target 
% being JavaScript, as Blockly itself is written in JavaScript and hosted in a web browser.


Two goals: 
\begin{itemize}
    \item education about coding; 
    \item education about domains/APIs;
\end{itemize}

% https://www.khanacademy.org/computing/computer-programming/programming/drawing-basics/p/making-drawings-with-code
% http://tryruby.org/levels/1/challenges/0

[goal: bring simplicity of blocky to more domains, to leverage all the JavaSript]

TypeScript (\url{www.typescriptlang.org}) is superset of the JavaScript language that is gradually typed, 
meaning that types may optionally be added to JavaScript code to provide for a more productive programming 
and debugging experience.  Many JavaScript frameworks provide TypeScript declaration files, as
can be found at the Definitely Typed GitHub repo.~\footnote{\url{https://github.com/DefinitelyTyped/DefinitelyTyped}}.

[need a running example]

The contribution of this paper is to describe a mapping from TypeScript annotated 
JavaScript APIs to Blockly that greatly simplifies 
the process of making existing JavaScript frameworks and libraries available via Blockly.

It is not our goal to map every TypeScript programming construct into Blockly.
For example, a one-to-one mapping of the TypeScript abstract syntax tree (AST) 
nodes to blocks would provide no abstraction, exposing the beginner to a visual 
representation ofTypeScript in its full glory.

Rather, it is support common programming paradigms with a simple mapping
from TypeScript to blocks and back.  This mapping may, in fact, be many-to-one, 
abstracting over multiple TypeScript elements to provide a single block abstraction.
This supports Blockly's goal to provide a simplified programming experience 
with higher-level abstractions. 

The challenge we face is provide a sufficiently rich mapping to handle the
many different API paradigms that JavaScript can support, while
also simplifying the APIs so that they are accessible from Blockly. 

\subsection{Example}

Below is example of a TypeScript function \emph{onButton},
with parameters $b$ and $f$,
that represents a common JavaScript pattern of registering an event
handler ($f$) to be executed when some event occurs (in this case, 
a press of button $A$ or button $B$, as specified in the enumeration
\emph{Button}):
\begin{lstlisting}
enum Button { A, B };
function onButton(b:  Button, 
                  f: () => void): void { }
\end{lstlisting}
The function is explicitly typed using TypeScript's support for
type annotations: each parameter has a type, which follows the colon;
furthermore, the return type of the function also is specified as void
via annotation. 

This function gives rise to a 
block $B$ named ``onButton'', shown in Figure~\ref{fig:onButton}
that represents a \emph{call} to the function
(later sections described how attributes in the comment preceding
the function allow the look and feel of the block to be customized).

The function definition and its types define the block's two \emph{inputs}:
\begin{itemize}
\item  a \emph{value} input corresponding to the parameter
$b$ which allows selection of one of two 
possible parameter values, either ``A'' or ``B'', corresponding to the
the two possible values for the enumeration \emph{Button} ---
while an enumeration is realized as a JavaScript number, a 
Blockly field selector for this parameter only allows
the user to choose either ``A'' or ``B'' (0 or 1);
\item a \emph{statement} input corresponding to the parameter $f$
that allows the user to drag statement blocks inside of block $B$
to define the body of the lambda function passed to $f$.
\end{itemize}
The resulting block allows user to instantiate the function call's
parameter values without the necessity to explicitly create a 
lambda function.  Let $S$ be the sequence of TypeScript statements
corresponding to the blocks of $B$'s statement input and XYZ. 

[TODO: show the block and the resulting TypeScript call]



\subsection{Overview}



% Nonetheless, there are features of the TypeScript language that one usually
% finds lacking in Blockly instantations, such as explicit use of lexical scoping,


% \begin{itemize}
%   \item lexical scoping (via let);
%   \item functions as values (also nested functions -> closures);
%   \item classes with single inheritance, methods, objects
%   \item arrays (lists);
%   \item object destructuring
% \end{itemize}

\section{Blockly Overview}
\label{sec:blockly}

Not suprisingly, the core abstraction of Blockly is the \emph{Block},
which is used to represent statements, expressions, values and variables.  
Blocks have \emph{connectors} that allow them to be sequenced
horizontally or vertically in space.  In the default Blockly
layout, vertically sequenced blocks
represent program statements, while horizontally
sequenced blocks represent program expressions/values.
Blockly also supports the nesting of blocks, and so a collection of
blocks naturally represents an abstract syntax tree. 

While Blockly does attach a loose meaning to blocks,
their actual semantics given to blocks is entirely up to the Blockly developer.
We use the TypeScript language (and its type system) to 
give a more precise meaning to blocks. 

\subsection{Block connections}

Block connectors, inputs and output:
\begin{itemize}
% A block with a previous connector cannot have an output connector, and vice versa. 
% The term statement block refers to a block with no value output. 

% nextStatement and previousStatement connections can be typed
% but this feature is not utilized by standard blocks.

\item \emph{previous and next connectors}, both optional, allow a block to be vertically sequenced - Blockly
  enforces that a block with a previous connector cannnot have an output connector; according
  to Blockly documentation, ``a statement block will usually have both a previous connection and 
  a next connection'';

\item a block may have a single \emph{output connector}, 
      which appears as a male jigsaw connector on a block's left-side; according to 
      Blockly documentation, ``Blocks with an output are usually called value blocks'';

\item a block may have multiple \emph{inputs}, which can appears as holes in a block
      or female jigsaw connectors on a block's right-side, about which we'll say more below. 
\end{itemize}

Block inputs come in three basic forms:
\begin{enumerate}
  \item \emph{fields}, which represent terminals (constants, literals, variables);
  \item \emph{value inputs} (receives value from output block) - value inputs
      and value blocks allow one to create expression trees;
  \item \emph{statement inputs}, allow one to create statement trees;
\end{enumerate}

\subsection{Blockly type checking}

\subsection{Blockly conventions (JavaScript)}
\begin{itemize}
  \item no variable declarations (all variables global, except for);
  \item dynamic typing;
  \item expressions can't be put where statement expected 
  - use an assignment to a dummy variable to make expression into statement);
  \item loops and local scope
\end{itemize}
\input{ts}
% blockId is a constant, unique id for the block. This id is serialized in block code so changing it will break your users.
% block contains the syntax to build the block structure (more below).
% Other optional attributes can also be used:

% blockExternalInputs= forces External Inputs rendering
% advanced=true causes this block to be placed under the parent category’s “More…” subcategory. Useful for hiding advanced or rarely-used blocks by default
% Block syntax

\section{TypeScript to Blockly}

Types and namespaces are an integral part of the TypeScript language that
\MC uses to define the mapping from TypeScript to Blockly. If you aren't
familiar with TypeScript, please review the 
\href{https://www.typescriptlang.org/docs/handbook/basic-types.html}{TypeScript Handbook}.

\subsection{TypeScript Declaration File}

Internally, 
\MC uses a TypeScript declaration file (*.d.ts) to define
the mapping to custom blocks,
using specialized comments that further parameterize the mapping
defined by the declarations.  
The declaration file, which contains no implementation (definitions),
generally is automatically generated from a TypeScript implementation
file or files (*.ts) by a preprocessing step of the \MC build. 
The text below pertains to both kinds of files.  However, it is good to remember
that it is solely the TypeScript declarations (not the particulars of how
the declarations are implemented) that define the mapping to custom blocks. 

For a simple example of a declaration file, see
\url{https://github.com/Microsoft/pxt-sample/blob/master/libs/core/sims.d.ts}.


\subsection{Namespaces, Functions, and Primitive Values}

The mapping of TypeScript to Blockly is straighforward for namespaces,
functions and basic primitive values (such as boolean, number and string).
The mapping utility recognizes the special comment `//\%' before a function
definition $f$ to indicate that the function should be exported into the 
set of user-facing TypeScript APIs (but without a corresponding custom block). The
addition of the attribute `block` to the comment indicates that a custom
block should be generated for the exported TypeScript function.  In this way, 
\MC provides two APIs from a single function definition: one is the
TypeScript function $f$ itself; the second is a custom block, which represents
a potentially simplified call to $f$ (not using all the parameters provided by
$f$, for example).

In TypeScript, a set of related definitions may be grouped in a namespace.
These namespaces map to Toolbox categories in Blockly;
in particular, each top-level namespace is used to populate a category 
in the toolbox. 

As seen in the introduction, 
the general representation of a custom block is through the declaration/definition
of a TypeScript function $f$.  In particular, the custom block for $f$ usually
corresponds to a call (invocation) of function $f$, where the block's inputs
are transformed to parameters of function $f$.

\subsection{Function Typing}

The declared type of a function informs the kind of block that will be created:
\begin{itemize}

  \item a void return type for a function means the custom block will be a statement
  block; otherwise, it will be an expression block;
  \item functions return a single value;
  \item parameters: optional parameters with default value; rest; 
  \item handler: (top level block)
\end{itemize}

% Block syntax

% The block attribute specifies how the parameters of the function will be organized to create the block.

% block = field, { '|' field }
% field := string
%     | string `%` parameter [ `=` type ]
% parameter = string
% type = string
% each field is mapped to a field in the block editor
% the function parameter are mapped in order to %parameter argument. 

% The loader automatically builds a mapping between the block field names and the function names.
% the block will automatically switch to external inputs when enough parameters are detected
% A block type =type can be specified optionally for each parameter. It will be used to populate the shadow type.
% Supported types

\subsection{Enumerations}

\subsection{Objects and Object Destructuring}
API represented by
\begin{itemize}
  \item set of classes, with
  \item constructors
  \item methods/fields
\end{itemize}

\subsection{Event Handlers and Callbacks}

% \item event-handlers via a callback function (callbacks only have parameters with primitive values);

A function $f$ that has an argument $g$ of function type (in last position) will have
that function argument (callback) converted into a statement input of $B(f)$.

If the callback $g$ has parameters, then
the best way to map that pattern to the blocks is by modifying
$g$ to have a single parameter with a class type that has the
various parameters as fields.
For example:

% export class ArgumentClass {
%     argumentA: number;
%     argumentB: string;
% }

% //% mutate=objectdestructuring
% //% mutateText="My Arguments"
% //% mutateDefaults="argumentA;argumentA,argumentB"
% // ...
% export function addSomeEventHandler((a: ArgumentClass) => void) { };

% In the above example, setting mutate=objectdestructuring will cause this API 
% to use Blockly “mutators” to let users change what parameters appear in the blocks. 
% Each parameter will be given an optional variable field in the block that defines a 
% variable that can be used within the callback. The variable fields compile to object
% destructuring in the TypeScript code. For example:

% addSomeEventHandler(({argumentA, argumentB}) => {

% })

% For an example of this pattern in action, see the radio.onDataPacketReceived block in the microbit target.

% In some cases it can be useful to change the runtime behavior of the API based on the properties 
% selected by the user. To enable that behavior, create an enum with entries that have the same names 
% as the argument object’s properties and add an extra parameter taking in an enum array to the API. 
% For example:

% export class ArgumentClass {
%     argumentA: number;
%     argumentB: string;
% }

% enum ArgNames {
%     argumentA,
%     argumentB
% }

% //% mutate=objectdestructuring
% //% mutateText="My Arguments"
% //% mutateDefaults="argumentA;argumentA,argumentB"
% //% mutatePropertyEnum="argNames"
% // ...
% export function addSomeEventHandler(args: ArgNames[], (a: ArgumentClass) => void) { };
% Note the mutatePropertyEnum attribute added to the comment annotations. The block for this API will look the same as the previous example but the compiled code will also include the arguments passed:

% addSomeEventHandler([ArgNames.argumentA, ArgNames.argumentB], ({argumentA, argumentB}) => {

% })
% The other attributes related to object destructuring mutators include:

% mutateText - defines the text that appears in the top block of the Blockly mutator dialog (the dialog that appears when you click the blue gear)
% mutateDefaults - defines the versions of this block that should appear in the toolbox. Block definitions are separated by semicolons and property names should be separated by commas


\section{Language}
\subsection{Lexical Scoping and Local Variables}

% to be implemented
% in most blockly programs, variables are used without being declared
% we use the block structure to determine where to place declaration
% given a set of uses U(X) of variable X, we take the least common ancestor
% of U(X) and place the declaration for X there when translating from
% blockly to typescript

% this means we won't preserve
\input{checking}

%% Bibliography
\bibliography{paper}

\end{document}

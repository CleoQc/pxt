\section{The \CO Runtime}
\label{sec:codal}

\CO is a lightweight, object-oriented, componentized C++ runtime for microcontrollers designed to provide an efficient abstraction layer for higher level languages, such as JavaScript. \CO has five key elements:

\begin{enumerate}
\item \emph{a unified eventing subsystem} (common to all components) that provides a mechanism to map asynchronous hardware and software events to event handlers;
\item \emph{a non-preemptive fiber scheduler} that enables concurrency while minimizing the need for resource locking primitives;
\item \emph{a simple memory management system} based on reference counting to provide a basis for managed types;
\item \emph{a set of drivers}, that abstract microcontroller hardware components into higher level software components, each presented as a C++ class;
\item \emph{a parameterized object model} composed from these components that represents a physical device.
\end{enumerate}
% TODO: add Fork on block

\subsection{Message Bus and Events}

\CO offers a simple yet powerful model for handling hardware or user defined events. Events are modeled as a tuple of two integer values - specifying an \emph{id} (namespace) and a \emph{value}.
Typically, an id correlates to a specific software component, which may be as simple as a button or something more complex as a wireless network interface. The value relates to a specific event that is unique within the id namespace. All events pass through the \CO MessageBus. Application developers can then \emph{listen} to events on this bus, by defining a C/C++ function to be invoked when an event is raised. Events can be raised at any time simply by creating an \emph{Event} C++ object, which then invokes the event handlers of any registered listeners.

Continuing the example of detecting the brightness of a room used in Section~\ref{sec:shim-gen}, a listen call to the MessageBus with a component ID of the light sensor and a threshold event is the underlying mechanism enabled by the runtime, as illustrated by the equivalent C++ code snippet in Figure~\ref{fig:messageBus}.

\begin{figure}[t]
    \begin{lstlisting}
    #include "CircuitPlayground.h"
    CircuitPlayground cplay;

    void onBright() { // user defined code }

    int main() {
        cplay.messageBus.listen(ID_LIGHT_SENSOR, LIGHT_THRESHOLD_HIGH, onBright);
    }
    \end{lstlisting}
    \caption{\label{fig:messageBus}Example of the \CO MessageBus.}
\end{figure}

Unlike simple function pointers, \CO event handlers can be parameterized by the event listener to provide decoupling between the context of the code raising the event. The receiver of an event can choose to either receive an event in context of the fiber that created it or can be decoupled and executed via an Asynchronous Procedure Call (APC). The former provides performance, while the latter provides decoupling of low level code (that may be executing, say, in an interrupt context) from user code. Each event handler may also define a threading model, so they can be reentrant or run-to-completion, depending upon the semantics required.

\subsection{Fiber Scheduler}
\CO provides a \emph{non-preemptive} fiber scheduler with asynchronous semantics and a power efficient implementation. \CO fibers can be created at any time but will only be descheduled as a result of an explicit call to yield(), sleep() or wait\_for\_event() on the MessageBus. The latter enables condition synchronization between fibers through a wait/notify mechanism. A round-robin approach is used to schedule runnable fibers. If at any time all fibers are descheduled, the MCU hardware is placed into a power efficient sleep state.

The \CO scheduler makes use of two novel mechanisms to optimize for MCU hardware. Firstly, \CO adopts a stack paging approach to fiber management. MCUs do not support virtual memory and are heavily RAM constrained, but relatively cycle rich. Therefore, instead of overprovisioning stack memory for each fiber (which would waste valuable RAM), we instead dynamically allocate stack memory from heap space as necessary and copy the physical stack into this space at the point at which a fiber is descheduled (and similarly restored when a fiber is scheduled). This copy operation clearly incurs a small CPU overhead, but brings greater benefits of RAM efficiency - especially given that MCU stack sizes are typically quite small (\textasciitilde200 bytes is typical).

Secondly, the \CO scheduler supports transparent APCs. Any function can be \emph{invoked} as an APC. Conceptually, this is equivalent to calling the given function in its own fiber. However, the \CO runtime provides a common-case transparent optimization for APCs we call \emph{fork-on-block} - whereby a fiber will only be created at the point at which the given function attempts a blocking operation such as sleep() or wait\_for\_event(). Functions which do not block therefore do not incur all of the context switch overhead.

When invoking an APC, the scheduler snapshots the current processor context and stack pointer (but not the whole stack). If the scheduler is re-entered before the APC completes, a new fiber context is created at the point of descheduling, and placed on the appropriate wait queue. The previously stored context is then restored, and execution continues from the point at which the APC was first invoked. This mechanism provides a potentially high RAM savings for processing of MessageBus event handlers in particular.

\CON's scheduling and eventing models are shared by both high and low level languages, and therefore handled uniformly. As a result, when a foreign function call is mapped to C++, that C++ function is capable of blocking the calling fiber without infringing on the concurrency model of the higher level language. This enables, for example, a C++ device driver to block a JavaScript program when awaiting data without changing the behavior of other JavaScript code acting asynchronously (as in Figure~\ref{fig:screenSnap}).

%To provide unity, \CO natively supports continuations in two ways. Firstly, through asynchronous procedure calls that may occur at any point in the future, where a condition and callback function are given (Figure~\ref{fig:messageBus}). Secondly, through the

\subsection{Memory Management}
\CO implements its own lightweight heap allocator, introducing reentrant versions of the libc malloc family of functions to permit universal access to heap memory in user or interrupt code. The heap allocator is flexible and reconfigurable, allowing the specification of multiple heaps across memory and it is optimized for repeat allocations of memory blocks that are commonplace in embedded systems.

\CO also makes use of simple managed types, built using C++ reference counting mechanisms. C++ classes are provided for common types such as strings, images, and data buffers. A generic base class is also provided for the creation of other managed types. This simple approach brings the benefits of greater memory safety for application code, but with the expense of suffering from the issues related to circular references. We take the view that such scenarios are rare in MCU applications, justifying this approach over a more complex garbage collection scheme and its overhead.

\subsection{Device Driver Components}
\CO drivers abstract away the complexities of the underlying hardware into reusable, extensible, easy-to-use components. For every hardware component there is a corresponding software component that encapsulates its behavior in a C++ object. \CO has three types of drivers:
\begin{enumerate}
    \item A hardware agnostic abstract specification of a driver model (e.g. a Button, or an Accelerometer). This is provided as a C++ base class.
    \item The concrete implementation of the abstract driver model, which is typically hardware specific. This is implemented as a subclass of a driver model, such as a LIS3DH accelerometer, as manufactured by ST Microelectronics.
    \item A high level driver that relies only on the interfaces specified in a driver model (e.g. a gesture recognizer based on an Accelerometer model).
\end{enumerate}

This approach brings the benefits of abstraction and re\-usability to \CON, without losing the hardware specific benefits seen in flat abstraction models where every MCU is made to look the same, even though their capabilities are different. This can be seen in the Arduino and mbed APIs, for example.

Finally, we group together the components of a physical device to form a \emph{device model}. This is a singleton C++ class that, through composition of device driver components, provides a configured representation of the capabilities of a device. Such a model provides for an elegant OO API for programming a device and a static representation that forms an ideal target for the \MC linker to bind high level STS interfaces to low level optimized code.

% An example device model for the CPX is shown in Figure~\ref{fig:codalDeviceModel} for reference.

 \MC is further supported by an annotated C++ library (\emph{\MC wrappers}) defining the mapping from \CO to TypeScript and Blockly. The use of \MC wrappers ensures that different \MC targets that use \CO share a common TypeScript and block API vocabulary.\footnote{See \url{https://github.com/microsoft/pxt-common-packages}}

% \begin{figure}
% % TODO: shrink width
% \begin{lstlisting}
% class CircuitPlayground : public CodalDevice {
%   public:
%     MessageBus                  messageBus;
%     CPlayTimer                  timer;
%     SAMD21Serial                serial;
%     CircuitPlaygroundIO         io;
%     Button                      buttonA;
%     SAMD21I2C                   i2c;
%     LIS3DH                      accelerometer;
%     NonLinearAnalogSensor       thermometer;
%     AnalogSensor                lightSensor;
%     ...
% \end{lstlisting}
% \caption{\label{fig:codalDeviceModel}Device Model for the Adafruit CPX}
% \end{figure}



